\documentclass[onecolumn, draftclsnofoot,10pt, compsoc]{IEEEtran}
\usepackage{graphicx}
\usepackage{url}
\usepackage{setspace}

\usepackage{geometry}
\geometry{textheight=9.5in, textwidth=7in}

% 1. Fill in these details
\def \CapstoneTeamName{		Project LOOM}
\def \CapstoneTeamNumber{		36}
\def \GroupMemberOne{			William Selbie}
\def \GroupMemberTwo{			Trevor Swope}
\def \GroupMemberThree{			Luke Goertzen}
\def \CapstoneProjectName{		Project LOOM | Design an Internet of Things Rapid Prototyping System.}
\def \CapstoneSponsorCompany{	}
\def \CapstoneSponsorPerson{		Chet Udell}

% 2. Uncomment the appropriate line below so that the document type works
\def \DocType{		Problem Statement
				%Requirements Document
				%Technology Review
				%Design Document
				%Progress Report
				}
			
\newcommand{\NameSigPair}[1]{\par
\makebox[2.75in][r]{#1} \hfil 	\makebox[3.25in]{\makebox[2.25in]{\hrulefill} \hfill		\makebox[.75in]{\hrulefill}}
\par\vspace{-12pt} \textit{\tiny\noindent
\makebox[2.75in]{} \hfil		\makebox[3.25in]{\makebox[2.25in][r]{Signature} \hfill	\makebox[.75in][r]{Date}}}}
% 3. If the document is not to be signed, uncomment the RENEWcommand below
\renewcommand{\NameSigPair}[1]{#1}

%%%%%%%%%%%%%%%%%%%%%%%%%%%%%%%%%%%%%%%
\begin{document}
\begin{titlepage}
    \pagenumbering{gobble}
    \begin{singlespace}
    	%\includegraphics[height=4cm]{coe_v_spot1}
        \hfill 
        % 4. If you have a logo, use this includegraphics command to put it on the coversheet.
        %\includegraphics[height=4cm]{CompanyLogo}   
        \par\vspace{.2in}
        \centering
        \scshape{
            \huge CS Capstone \DocType \par
            {\large\today}\par
            \vspace{.5in}
            \textbf{\Huge\CapstoneProjectName}\par
            \vfill
            {\large Prepared for}\par
            \Huge \CapstoneSponsorCompany\par
            \vspace{5pt}
            {\Large\NameSigPair{\CapstoneSponsorPerson}\par}
            {\large Prepared by }\par
            Group\CapstoneTeamNumber\par
            % 5. comment out the line below this one if you do not wish to name your team
            \CapstoneTeamName\par 
            \vspace{5pt}
            {\Large
                \NameSigPair{\GroupMemberOne}\par
                \NameSigPair{\GroupMemberTwo}\par
                \NameSigPair{\GroupMemberThree}\par
            }
            \vspace{20pt}
        }
        \begin{abstract}
        % 6. Fill in your abstract    
        	Our project involves creating a rapid prototyping system in the field of Internet of Things that can be used by those with minimal technical experience, with specific focus on use in the field of agriculture.
			This will take the form of modular hardware kits that can be programmed using a graphical interface so that the user never has to write a line of code. The reason that this is so useful is because of the current barriers to entry for Internet of Things type solutions greatly restrict the quantity of solutions being proposed. At the same time, Internet of Things related ideas have the potential to solve a large quantity of problems, but a lot of the population lacks the technical skills to enact these solutions even though they have the creativity to come up with them. 
			By giving those without technical experience the ability to create and build their own solutions we are removing one of the barriers to entry to creating Internet of Things type solutions, resulting, in theory, in a greater number of problems being solved. 
        \end{abstract}     
    \end{singlespace}
\end{titlepage}
\newpage
\pagenumbering{arabic}
\tableofcontents
% 7. uncomment this (if applicable). Consider adding a page break.
%\listoffigures
%\listoftables
\clearpage

% 8. now you write!
\section{Problem Statement}
In the current rapidly advancing state of technology, specifically in the area of Internet of Things, everyone should be able to manifest their ideas and creativity and take advantage of all of the features that are being built into everyday appliances and devices.
Unfortunately, this is not the case, the barriers to entry in terms of knowledge and experience are typically quite high, which is restricting the quantity of ideas that can take advantage of the Internet of Things.
Many products that the Internet of Things concept would take advantage of require at the least some programming experience, and typically a fair bit more than that in order to setup and utilize currently available technology.
The issue here is that not a significant portion of the population has enough experience to hurdle the barriers to entry, while in contrast a large fraction of the population likely has ideas that could take advantage of the interconnectedness of such a variety of objects.
Project LOOM is a little more focused than general Internet of Things, specifically in the field of agriculture.
For instance, farmers and those growing crops could benefit massively from having smart devices monitor things like soil moisture content, nitrogen content, and crop growth rates, and then having analysis performed on this information. 
Unfortunately, the ordeal of setting up and programming the behavior of such a system makes it infeasible, despite the benefits that could come of it. 
Ultimately, Internet of Things has the capability to drastically improve the everyday lives of all kinds of people because of how inherently flexible and far reaching it is in nature, with the only limits generally being cost and creativity. This means increasing the pool from which to draw creativity has the potential to exponentially increase the number of solutions implemented through Internet of Things.
\section{Proposed Solution}
The solution to this problem is simple in concept though complicated in execution: design prebuilt kits with defined and explicitly listed capabilities that the customer can then program using a user-friendly graphical interface that requires no experience to use.
By pre assembling the kits and explicitly listing its capabilities, the customer can simply look through each kit’s potential uses and then determine based on already performed analysis whether or not a specific kit will suit their task.
Once the hardware is selected, the user can then program it with a graphical interface that doesn’t require knowledge but still amply rewards it with a system that is easy to use while still providing scalability and flexibility.
Given capabilities in terms of inputs, outputs, and processing that a kit can make use of, it should be easy for users to determine which kit, if any, is appropriate for their task. 
For instance, let’s say a farmer would like to have a smart sprinkler system turn on for a certain period of time each day based on surrounding soil moisture content as well as the current temperature.
Then this farmer would be looking for a kit which is capable of measuring both temperature and moisture content in soil, as well as relaying that information to an actuator of some kind that can analyze the given data before deciding to either turn the sprinkler on or off.
However, this would just be one of the applications of that kit, as it would also be capable of receiving more types of data inputs, manipulating and analyzing data in different ways, and taking different actions based on that information.
Perhaps this kit is also capable of measuring the amount of sunlight or cloud coverage the surrounding area is receiving.
The farmer could use the kit so that upon reaching a certain threshold of sunlight, it will rotate the sprinkler to point to a location that doesn’t have as much sun to minimize the amount of water that evaporates when watering crops.
The point being that with the given inputs and outputs that the kit is capable of receiving and sending, there’s a variety of potential uses for it depending upon the needs of the client. 
\section{Performance Metrics}
The most crucial performance metric will be a prototype proof of concept: a modular kit capable of being configured to solve multiple problems and be used for a variety of tasks without the need for any soldering, programming experience, or generally technical abilities not held by the average person.
It’s been mentioned several times that the kits should be usable by those with no programming or electrical engineering experience, but it’s not within the scope of the capstone project for us to go through user testing in order to see just how “usable” the kits are for non-technical people.
Another metric will be having a working software tool that allows users to program kits graphically so that the user will not need to write a single line of code throughout the creation of the solution.
The software tool has to be able use all the data that can be measured by sensors in the kit, perform some kind of digital logic based on data, let the user alter the flow and frequency of the program, and send analyzed data or commands to actuators to make things happen in the real world. 

\end{document}
