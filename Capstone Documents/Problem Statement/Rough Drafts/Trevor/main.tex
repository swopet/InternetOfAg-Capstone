\documentclass[onecolumn, draftclsnofoot,10pt, compsoc]{IEEEtran}
\usepackage{graphicx}
\usepackage{url}
\usepackage{setspace}

\usepackage{geometry}
\geometry{textheight=9.5in, textwidth=7in}

% 1. Fill in these details
\def \CapstoneTeamName{		Project LOOM}
\def \CapstoneTeamNumber{	    36}
\def \GroupMemberOne{			Trevor Swope}
\def \GroupMemberTwo{			William Selbie}
\def \GroupMemberThree{			Luke Goertzen}
\def \CapstoneProjectName{		Project LOOM}
\def \CapstoneSponsorCompany{	Oregon State University}
\def \CapstoneSponsorPerson{		Chet Udell}

% 2. Uncomment the appropriate line below so that the document type works
\def \DocType{		Problem Statement
				%Requirements Document
				%Technology Review
				%Design Document
				%Progress Report
				}
			
\newcommand{\NameSigPair}[1]{\par
\makebox[2.75in][r]{#1} \hfil 	\makebox[3.25in]{\makebox[2.25in]{\hrulefill} \hfill		\makebox[.75in]{\hrulefill}}
\par\vspace{-12pt} \textit{\tiny\noindent
\makebox[2.75in]{} \hfil		\makebox[3.25in]{\makebox[2.25in][r]{Signature} \hfill	\makebox[.75in][r]{Date}}}}
% 3. If the document is not to be signed, uncomment the RENEWcommand below
%\renewcommand{\NameSigPair}[1]{#1}

%%%%%%%%%%%%%%%%%%%%%%%%%%%%%%%%%%%%%%%
\begin{document}
\begin{titlepage}
    \pagenumbering{gobble}
    \begin{singlespace}
    	%\includegraphics[height=4cm]{coe_v_spot1}
        \hfill 
        % 4. If you have a logo, use this includegraphics command to put it on the coversheet.
        %\includegraphics[height=4cm]{CompanyLogo}   
        \par\vspace{.2in}
        \centering
        \scshape{
            \huge CS Capstone \DocType \par
            {\large\today}\par
            \vspace{.5in}
            \textbf{\Huge\CapstoneProjectName}\par
            \vfill
            {\large Prepared for}\par
            \Huge \CapstoneSponsorCompany\par
            \vspace{5pt}
            {\Large\NameSigPair{\CapstoneSponsorPerson}\par}
            {\large Prepared by }\par
            Group\CapstoneTeamNumber\par
            % 5. comment out the line below this one if you do not wish to name your team
            %\CapstoneTeamName\par 
            \vspace{5pt}
            {\Large
                \NameSigPair{\GroupMemberOne}\par
                %\NameSigPair{\GroupMemberTwo}\par
                %\NameSigPair{\GroupMemberThree}\par
            }
            \vspace{20pt}
        }
        \begin{abstract}
        % 6. Fill in your abstract    
        	Our project will entail creating a rapid prototyping system for Internet of Things devices, for use by individuals with limited to no technical experience.
    Internet of Things devices have high applicability to many different fields; in our case, we are focused on agricultural monitoring uses, but in general there is a high technical barrier to people setting up Internet of Things projects, many of which are fundamentally similar.
    Our project aims to capture that fundamental similarity by creating a modular system of building blocks which can be physically put together and represented in a user interface through which the user generates the firmware for the system without writing a single line of code themselves.
    If successful, this project will allow the integration of Internet of Things devices and concepts into numerous different real-world applications and educational programs.

        \end{abstract}     
    \end{singlespace}
\end{titlepage}
\newpage
\pagenumbering{arabic}
%\tableofcontents
% 7. uncomment this (if applicable). Consider adding a page break.
%\listoffigures
%\listoftables
%\clearpage

% 8. now you write!
\section{Problem Statement}
Project LOOM aims to create a plug-and-play suite of interconnected Internet Of Things chips that can be used in a modular fashion to create complex systems from simple building blocks.
There is a large learning curve to developing simple solutions for problems in the realm of the Internet Of Things, a relatively new concept that entails any interconnected embedded systems.
The idea behind Project LOOM is that we, the developers, create a suite of building blocks, in the form of various chips and shields with different capabilities, and given a spec, a user is able to generate the necessary firmware to make those chips work for their purpose without writing a single line of code.
We aim to do this by creating, along with the building blocks themselves, a user interface for creating and setting up the system, as well as for monitoring the data that is collected and sending remote signals as needed. \\
To monitor the data that is collected and control the system from afar, the chips need to be capable of communicating wirelessly through one or more of a few flavors of gateway that we will use: WiFi, LoRa, and Nordic RF.
The suite of sensors we use will have a primarily agricultural focus (as that has traditionally been the focus of the Open Sensing Lab), but will be extensible to use in numerous other contexts.
Sensors we use will hopefully include temperature, moisture, light, motion sensing and various output behaviors.
Hopefully, we will be able to create a final product that is accessible and useful for many different parties, expanding the realm of possibilities of the Internet of Things.

\section{Proposed Solution}
Our proposed solution is to create a user interface similar to or directly based on Labview, which essentially boils down to drag-and-drop programming of the firmware.
A user will create a module associated with the system in Labview, which will break down into boxes that represent the chips, without input and output behaviors.
These behaviors will be defined by drop-down menus with values for the user to fill in.
Once these behaviors are determined by the user, the program will use preprocessor directives to compile the appropriate firmware for the system.
We will be using 3 types of wireless communication: WiFi, low-frequency radar (LoRa), and Nordic RF.
We will likely use ethernet for early prototyping while the gateway part of the project is being developed.
Shields we intend to use for the chips include relays, RTC+SD drivers, steppers and motor drivers, servo's, various sensors for temperature, moisture, movement, and light, and others.
To write the source code we will be using primarily the language of C and at least for the beginning, the Arduino integrated development environment.
For now, at least, security of these chips is out of the scope of our project, but in the future it will probably be a vital part to making the project viable in the real world, particularly in industrial contexts.

\section{Performance Metrics}
Our first metric is to have a working, streamlined prototype by the end of Fall Term 2017, entailing 3 different kinds of sensors, one flavor of gateway (probably ethernet), and a working demo of the user interface that allows faculty and industry guests to engage with the project.
This will require a working user interface and the least viable version of the source code for the firmware.
This source code will entail conditional pre-processor statements whose values are determined by the user. \\
At the end of the project, we want to be able to think of a possible use for the chipset and be able to implement it without writing any new code.
We can test this by physically testing the chips once our firmware is flashed onto them and demonstrating its behavior.
We hope to have a plethora of sensors and chips created that are compatible with our software.
These will include adapters and antennae capable of transmitting over both short and long distances for extended periods of time without frequent maintenance.

\end{document}