\documentclass[onecolumn, draftclsnofoot,10pt, compsoc]{IEEEtran}
\usepackage{graphicx}
\usepackage{url}
\usepackage{setspace}

\usepackage{geometry}
\geometry{textheight=9.5in, textwidth=7in}

% 1. Fill in these details
\def \CapstoneTeamName{		    \emph{Team Name}}
\def \CapstoneTeamNumber{		36}
\def \GroupMemberOne{			Trevor Swope}
\def \GroupMemberTwo{			William Selbie}
\def \GroupMemberThree{			Luke Goertzen}
\def \CapstoneProjectName{		Project LOOM}
\def \CapstoneSponsorCompany{	OSU | Openly Published Environmental Sensing Lab}
\def \CapstoneSponsorPerson{	Chet Udell}

% 2. Uncomment the appropriate line below so that the document type works
\def \DocType{		Problem Statement
				%Requirements Document
				%Technology Review
				%Design Document
				%Progress Report
				}
			
\newcommand{\NameSigPair}[1]{\par
\makebox[2.75in][r]{#1} \hfil 	\makebox[3.25in]{\makebox[2.25in]{\hrulefill} \hfill		\makebox[.75in]{\hrulefill}}
\par\vspace{-12pt} \textit{\tiny\noindent
\makebox[2.75in]{} \hfil		\makebox[3.25in]{\makebox[2.25in][r]{Signature} \hfill	\makebox[.75in][r]{Date}}}}
% 3. If the document is not to be signed, uncomment the RENEWcommand below
%\renewcommand{\NameSigPair}[1]{#1}

%%%%%%%%%%%%%%%%%%%%%%%%%%%%%%%%%%%%%%%
\begin{document}
\begin{titlepage}
    \pagenumbering{gobble}
    \begin{singlespace}
    	% \includegraphics[height=4cm]{coe_v_spot1}
        \hfill 
        % 4. If you have a logo, use this includegraphics command to put it on the coversheet.
        %\includegraphics[height=4cm]{CompanyLogo}   
        \par\vspace{.2in}
        \centering
        \scshape{
            \huge CS Capstone \DocType \par
            {\large\today}\par
            \vspace{.5in}
            \textbf{\Huge\CapstoneProjectName}\par
            \vfill
            {\large Prepared for}\par
            \Huge \CapstoneSponsorCompany\par
            \vspace{5pt}
            {\Large\NameSigPair{\CapstoneSponsorPerson}\par}
            {\large Prepared by }\par
            Group\CapstoneTeamNumber\par
            % 5. comment out the line below this one if you do not wish to name your team
            \CapstoneTeamName\par 
            \vspace{5pt}
            {\Large
                % \NameSigPair{\GroupMemberOne}\par
                % \NameSigPair{\GroupMemberTwo}\par
                \NameSigPair{\GroupMemberThree}\par
            }
            \vspace{20pt}
        }
        \begin{abstract}
        % 6. Fill in your abstract    
            The purpose of Project Loom is to design and create a system to be used for the rapid prototyping of Internet of Things ideas. The intent is for the system to be use modular kits of sensors that can be easily programmed in a straight-forward process without the need to do much electronics work or programming. As such, the project would allow those who are not necessarily technically literate to be able implement their own Internet of Things solutions without needing to purchase an existing, non-tailored system, or hiring a contractor to build one for them. This means that a greater number and wider variety of Internet of Things project can be developed easily by those in many fields, not just those in engineering and computer science based ones.
        \end{abstract}     
    \end{singlespace}
\end{titlepage}
\newpage
\pagenumbering{arabic}
\tableofcontents
% 7. uncomment this (if applicable). Consider adding a page break.
%\listoffigures
%\listoftables
\clearpage

% 8. now you write!
\section{Problem Definition and Description}
    The goal of Project LOOM is to address the limitations present in developing new Internet of Things networks and applications. The Internet of Things is simply the idea of a network of remotely connected devices that can be used for the purpose of data collection, or for remotely controlling and automating systems. Essentially, any number of relatively simple chips can be aggregated via the internet to create complex systems in the real world. One of the current restrictions that people and businesses face when seeking to develop a solution is that even for simple projects, there is a fairly high level of expertise required to implement it, which only becomes worse with increasingly complex systems. Anybody might have an idea for an Internet of Things project, but not everyone is familiar with the technology and programming required to set it up. Project LOOM is focused mostly towards agriculture but the concept can be extended with further sensors and actuators to address any field’s Internet of Things needs. 

\section{Proposed Solution}
    Project LOOM’s solution to the aforementioned problem is to abstract away much of the details and inner workings of the devices to present the user with a friendly interface to set up their idea. To do this, we will create pre-built modules with clear sensing or actuating functionality that can be used without the user needing to be familiar with the soldering or programming that went into making the devices. The modules are simply interchangeable, self-contained devices that can be connected to an Internet of Things network with minimal configuration. Logical combinations of modules can be grouped into kits, but separate sets could still be mixed and used with other sets without issue. \\
    The user should be able to use the devices based on their behavior, without necessarily knowing how we implemented said functionality. We will use a variety of shields (microcontroller extensions) to do this, ranging from motors and servos, to any variety of sensors. The communication between devices and a central computer (if any) will be wireless, using WiFi, LoRa, and Nordic RF - though wired ethernet will be used in some stages of the development process. We will provide an easy to use interface that allows the user to set up and configure their devices via a drag-and-drop interface, and then will be able to easily monitor or control that system via an app.

\section{Desired Outcomes}
    By separating the user from the more technical details, a wider variety of people would be able to design and use Internet of Things projects. Our goal with Project LOOM is to develop a suite of modules that we can use for proofs of concept for demonstration. More specifically, by the end of Fall term, we would like to have a proof of concept using at least three sensors or actuators, and connected via ethernet. Further development will move to wireless communication, more modules, and general improvements. 

\section{Performance Metrics}
    Our primary metric will be our proofs of concept, both at the end of Fall term, and at the end of the year for expo. By the end of the year, we would like to have a variety of modules to work with and be able to actually develop our proof of concept as a user would - that is, via the user interface that we will have designed. While we might have people such as fellow students test or give feedback on how easy our platform is to use, formal case studies are beyond the scope of the senior capstone project.
    
\end{document}